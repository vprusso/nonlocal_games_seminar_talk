\documentclass[tikz]{standalone}

%-----------------------------------------------------------------------------%
% Fonts and symbols:
%-----------------------------------------------------------------------------%

\usepackage{mathpazo}
\usepackage{amsfonts}
\usepackage{amsmath,relsize}
\usepackage{amssymb}
\usepackage{latexsym}

\usepackage{eepic}
\usepackage{sectsty}
\usepackage{textcomp}
\usepackage[usenames,dvipsnames,svgnames,table]{xcolor}

\usepackage{listings}
\usepackage{cancel}
\usepackage{booktabs}
\usepackage{siunitx}

\usepackage{pgfplots}
\usepackage{pgf}
\usepackage{multicol}
\usepackage{tikz}
\usepackage{centernot}
\usepackage{filecontents}

\usepgfplotslibrary{patchplots} % LATEX and plain TEX
\usetikzlibrary{pgfplots.patchplots,snakes,arrows} % LATEX and plain TEX
\usetikzlibrary{decorations.markings}
\pgfplotsset{width=9cm,compat=1.5.1}
\usepackage[T1]{fontenc}
\usepackage{graphicx}

%-----------------------------------------------------------------------------%
% Margins and page layout:
%-----------------------------------------------------------------------------%

\usepackage[margin=1in]{geometry}
\usepackage{hyperref}
\hypersetup{pdfpagemode=UseNone}

%-----------------------------------------------------------------------------%
% Theorem-like environments:
%-----------------------------------------------------------------------------%

\usepackage{amsthm}
\newtheorem{theorem}{Theorem}[section]
\newtheorem{lemma}{Lemma}[section]
\newtheorem{prop}{Proposition}[section]
\newtheorem{cor}{Corollary}[section]
\newtheorem{claim}{Claim}[section]
\newtheorem{question}{Question}[section]
\newtheorem{result}{Result}[section]
\newtheorem{conjecture}{Conjecture}[section]
\theoremstyle{definition}
\newtheorem{definition}{Definition}[section]
\newtheorem{remark}{Remark}[section]
\newtheorem{example}{Example}[section]

%-----------------------------------------------------------------------------%
% Other packages:
%-----------------------------------------------------------------------------%

\usepackage{xcolor}
\usepackage{makeidx}
\usepackage{graphicx}
\usepackage{gastex}
\usepackage{listings}
\usepackage{tikz}
\usetikzlibrary{calc,matrix,positioning,arrows,decorations.pathreplacing,shapes.symbols,automata,chains,fit,shapes,er,petri,patterns}
\usetikzlibrary{decorations.pathmorphing}

%-----------------------------------------------------------------------------%
% Macros:
%-----------------------------------------------------------------------------%

\newcommand{\comment}[1]{\textcolor{blue}{%
  \begin{quote}\sf [*** #1 ***]\end{quote}}}

\newcommand{\tinyspace}{\mspace{1mu}}
\newcommand{\microspace}{\mspace{0.5mu}}
\newcommand{\op}[1]{\operatorname{#1}}
\newcommand{\tr}{\operatorname{Tr}}
\newcommand{\rank}{\operatorname{rank}}
\renewcommand{\int}{\operatorname{int}}
\renewcommand{\det}{\operatorname{Det}}
\renewcommand{\vec}{\operatorname{vec}}
\newcommand{\fid}{\operatorname{F}}
\newcommand{\im}{\operatorname{im}}
\newcommand{\reg}[1]{\mathsf{#1}}

\newcommand{\spans}{\textnormal{span}}
\newcommand{\proj}{\textnormal{proj}}

\renewcommand{\t}{{\scriptscriptstyle\mathsf{T}}}

\newcommand{\sA}{{\scriptscriptstyle\mkern-1mu\textup{A}}}
\newcommand{\sB}{{\scriptscriptstyle\mkern-0mu\textup{B}}}
\newcommand{\ssA}{{\scriptscriptstyle\mkern-1mu\textup{\miniscule{A}}}}
\newcommand{\ssB}{{\scriptscriptstyle\mkern-0mu\textup{\miniscule{B}}}}

\newcommand{\abs}[1]{\lvert #1 \rvert}
\newcommand{\bigabs}[1]{\bigl\lvert #1 \bigr\rvert}
\newcommand{\Bigabs}[1]{\Bigl\lvert #1 \Bigr\rvert}
\newcommand{\biggabs}[1]{\biggl\lvert #1 \biggr\rvert}
\newcommand{\Biggabs}[1]{\Biggl\lvert #1 \Biggr\rvert}

\newcommand{\ip}[2]{\langle #1 , #2\rangle}
\newcommand{\bigip}[2]{\bigl\langle #1, #2 \bigr\rangle}
\newcommand{\Bigip}[2]{\Bigl\langle #1, #2 \Bigr\rangle}
\newcommand{\biggip}[2]{\biggl\langle #1, #2 \biggr\rangle}
\newcommand{\Biggip}[2]{\Biggl\langle #1, #2 \Biggr\rangle}

\newcommand{\ceil}[1]{\lceil #1 \rceil}
\newcommand{\bigceil}[1]{\bigl\lceil #1 \bigr\rceil}
\newcommand{\Bigceil}[1]{\Bigl\lceil #1 \Bigr\rceil}
\newcommand{\biggceil}[1]{\biggl\lceil #1 \biggr\rceil}
\newcommand{\Biggceil}[1]{\Biggl\lceil #1 \Biggr\rceil}

\newcommand{\floor}[1]{\lfloor #1 \rfloor}
\newcommand{\bigfloor}[1]{\bigl\lfloor #1 \bigr\rfloor}
\newcommand{\Bigfloor}[1]{\Bigl\lfloor #1 \Bigr\rfloor}
\newcommand{\biggfloor}[1]{\biggl\lfloor #1 \biggr\rfloor}
\newcommand{\Biggfloor}[1]{\Biggl\lfloor #1 \Biggr\rfloor}

\newcommand{\norm}[1]{\lVert\tinyspace #1 \tinyspace\rVert}
\newcommand{\bignorm}[1]{\bigl\lVert\tinyspace #1 \tinyspace\bigr\rVert}
\newcommand{\Bignorm}[1]{\Bigl\lVert\tinyspace #1 \tinyspace\Bigr\rVert}
\newcommand{\biggnorm}[1]{\biggl\lVert\tinyspace #1 \tinyspace\biggr\rVert}
\newcommand{\Biggnorm}[1]{\Biggl\lVert\tinyspace #1 \tinyspace\Biggr\rVert}

\newcommand{\bigtriplenorm}[1]{
  \bigl\lvert\!\microspace\bigl\lvert\!\microspace\bigl\lvert #1 
  \bigr\rvert\!\microspace\bigr\rvert\!\microspace\bigr\rvert}

\newcommand{\Bigtriplenorm}[1]{
  \Bigl\lvert\!\microspace\Bigl\lvert\!\microspace\Bigl\lvert #1 
  \Bigr\rvert\!\microspace\Bigr\rvert\!\microspace\Bigr\rvert}

\newcommand{\biggtriplenorm}[1]{
  \biggl\lvert\!\microspace\biggl\lvert\!\microspace\biggl\lvert #1 
  \biggr\rvert\!\microspace\biggr\rvert\!\microspace\biggr\rvert}

\newcommand{\Biggtriplenorm}[1]{
  \Biggl\lvert\!\microspace\Biggl\lvert\!\microspace\Biggl\lvert #1 
  \Biggr\rvert\!\microspace\Biggr\rvert\!\microspace\Biggr\rvert}

\newcommand{\triplenorm}[1]{
  \left\lvert\!\microspace\left\lvert\!\microspace\left\lvert #1 
  \right\rvert\!\microspace\right\rvert\!\microspace\right\rvert}
\def\iso{\cong}
\newcommand{\defeq}{\triangleq}

\newcommand{\bm}[1]{
  \left[\microspace #1 \microspace \right]}


\newcommand{\ket}[1]{
  \lvert\microspace #1 \microspace \rangle}

\newcommand{\bigket}[1]{
  \bigl\lvert\microspace #1 \microspace \bigr\rangle}

\newcommand{\Bigket}[1]{
  \Bigl\lvert\microspace #1 \microspace \Bigr\rangle}

\newcommand{\biggket}[1]{
  \biggl\lvert\microspace #1 \microspace \biggr\rangle}

\newcommand{\Biggket}[1]{
  \Biggl\lvert\microspace #1 \microspace \Biggr\rangle}

\newcommand{\bra}[1]{
  \langle\microspace #1 \microspace \rvert}

\newcommand{\bigbra}[1]{
  \bigl\langle\microspace #1 \microspace \bigr\rvert}

\newcommand{\Bigbra}[1]{
  \Bigl\langle\microspace #1 \microspace \Bigr\rvert}

\newcommand{\biggbra}[1]{
  \biggl\langle\microspace #1 \microspace \biggr\rvert}

\newcommand{\Biggbra}[1]{
  \Biggl\langle\microspace #1 \microspace \Biggr\rvert}

\def\I{\mathbb{1}}
\def\G{\mathsf{G}}
\def\BB84{\mathsf{BB84}}
\def\MUB{\mathsf{MUB}}
\def\NPA{\mathsf{NPA}}
\def\NPAR{\mathsf{NPAR}}

\newcommand{\mh}{\textnormal{-}}

\def\MIP{\mathsf{MIP}}
\def\EMIP{\mathsf{EMIP}}


\newcommand{\setft}[1]{\mathrm{#1}}
\newcommand{\Density}{\setft{D}}
\newcommand{\Pos}{\setft{Pos}}
\newcommand{\Proj}{\setft{Proj}}
\newcommand{\Unitary}{\setft{U}}
\newcommand{\Herm}{\setft{Herm}}
\newcommand{\Lin}{\setft{L}}
\newcommand{\Channel}{\setft{C}}
\newcommand{\Trans}{\setft{T}}
\newcommand{\SepD}{\setft{SepD}}
\newcommand{\Sep}{\setft{Sep}}
\newcommand{\PPT}{\setft{PPT}}
\newcommand{\LOCC}{\setft{LOCC}}
\newcommand{\BPos}{\setft{Sep^*}}

% Game values
\newcommand{\unent}{\setft{\textnormal{unent}}}
\newcommand{\ent}{\setft{\textnormal{ent}}}
\newcommand{\val}{\setft{\textnormal{val}}}
\newcommand{\com}{\setft{\textnormal{com}}}
\newcommand{\sdp}{\setft{\textnormal{sdp}}}
\newcommand{\ns}{\setft{\textnormal{ns}}}

\newcommand{\negs}{\textnormal{-}}

\makeatletter
\providecommand*{\cupdot}{%
  \mathbin{%
    \mathpalette\@cupdot{}%
  }%
}
\newcommand*{\@cupdot}[2]{%
  \ooalign{%
    $\m@th#1\cup$\cr
    \hidewidth$\m@th#1\cdot$\hidewidth
  }%
}
\makeatother

\def\complex{\mathbb{C}}
\def\real{\mathbb{R}}
\def\natural{\mathbb{N}}
\def\integer{\mathbb{Z}}

\newenvironment{mylist}[1]{\begin{list}{}{
	\setlength{\leftmargin}{#1}
	\setlength{\rightmargin}{0mm}
	\setlength{\labelsep}{2mm}
	\setlength{\labelwidth}{8mm}
	\setlength{\itemsep}{0mm}}}
	{\end{list}}

\newenvironment{namedtheorem}[1]
	       {\begin{trivlist}\item {\bf #1.}\em}{\end{trivlist}}

\usepackage[affil-it]{authblk}

\def\X{\mathcal{X}}
\def\H{\mathcal{H}}
\def\Y{\mathcal{Y}}
\def\Z{\mathcal{Z}}
\def\W{\mathcal{W}}
\def\A{\mathcal{A}}
\def\B{\mathcal{B}}
\def\V{\mathcal{V}}
\def\U{\mathcal{U}}
\def\C{\mathcal{C}}
\def\D{\mathcal{D}}
\def\E{\mathcal{E}}
\def\F{\mathcal{F}}
\def\M{\mathcal{M}}
\def\N{\mathcal{N}}
\def\R{\mathcal{R}}
\def\Q{\mathcal{Q}}
\def\P{\mathcal{P}}
\def\S{\mathcal{S}}
\def\T{\mathcal{T}}
\def\K{\mathcal{K}}
\def\L{\mathcal{L}}

\newcommand{\class}[1]{\mathsf{#1}}

\newcommand{\w}{\mathsf{w}}

\def\NS{\mathcal{NS}}

\def\yes{\text{yes}}
\def\no{\text{no}}


\begin{document}

\tikzstyle{block} = [draw,fill=blue!20,minimum size=2em]
% diameter of semicircle used to indicate that two lines are not connected
\def\radius{1.8mm} 
\tikzstyle{branch}=[fill,shape=circle,minimum size=0pt,inner sep=0pt]
    % Define a style for shifting a coordinate upwards
    % Note the curly brackets around the coordinate.
    \tikzstyle{s}=[shift={(0mm,\radius)}]
    
\begin{center}
\begin{tikzpicture}[>=latex']

			\tikzstyle{Ref} = [shape=rectangle,rounded corners,draw=black!50,thick,fill=black!20,minimum height=10mm,minimum width=9mm,inner sep=1ex]
			\tikzstyle{Sig}=[shape=rectangle,rounded corners,dotted,draw=black!50,thick,fill=gray!15,minimum height=10mm,minimum width=10mm]
			\tikzstyle{AB}=[shape=rectangle,rounded corners,dotted,draw=black!50,thick,fill=gray!15,minimum height=15mm,minimum width=10mm]

			\node[Ref] (Ref-0-1) at (0,2.5) {\small $R_{0,1}$};			
			\node[Ref] (Ref-0-r) at (0,-2.5) {\small $R_{0,n}$};					
			
			\node[Ref] (Ref-1-0) at (6,2.5) {\small $R_{1,1}$};
			\node[Ref] (Ref-1-r) at (6,-2.5) {\small $R_{1,n}$};
						
			\node[AB] (A-1) at (3,2.2) {\small $A$};
			\node[AB] (B-1) at (3,-2.2) {\small $B$};
			
		  \node[Sig] (sig) at (-2, 0) {\small $\sigma$};
		  
		  \draw[-,thick] (-2,0.5) to (-2,4.5) {};
		  \draw[-,thick] (-2,4.5) to node[above] {$\reg{R}_1$} (6,4.5);
		  \draw[-,thick] (6,4.5) to (6,3) {};
		  
		  \draw[-,thick] (-2,-0.5) to (-2,-4.5) {};
		  \draw[-,thick] (-2,-4.5) to node[above] {$\reg{R}_n$} (6,-4.5);
		  \draw[-,thick] (6,-4.5) to (6,-3) {};

			\draw[-,thick] (-1,0.2) to (-1,4) {};
			\draw[-,thick] (-1,0.2) to (-1.5,0.2) {};
			\draw[-,thick] (-1,4) to node[above] {$\reg{X}$} (3,4);
			\draw[-,thick] (3,4) to (3,3);
			
			\draw[-,thick] (-1,-0.2) to (-1,-4) {};
			\draw[-,thick] (-1,-0.2) to (-1.5,-0.2) {};
			\draw[-,thick] (-1,-4) to node[above] {$\reg{Y}$} (3,-4);
			\draw[-,thick] (3,-4) to (3,-3);

			\node[] (elips-r) at (0,0) {\large{$\vdots$}};			  
			\node[] (elips-a) at (4.5,2.5) {\large{$\vdots$}};			  
			\node[] (elips-b) at (4.5,-2) {\large{$\vdots$}};			  
			\node[] (elips-m) at (6,0) {\large{$\vdots$}};			  
		  
		  \draw[double,thick] (0.5,2.8) to node[above] {$x_1$} (2.5,2.8);
		  \draw[double,thick] (3.5,2.8) to node[above] {$a_1$} (5.5,2.8);
		  \draw[double,thick] (0.5,-2.8) to node[above] {$y_n$} (2.5,-2.8);
		  \draw[double,thick] (3.5,-2.8) to node[above] {$b_n$} (5.5,-2.8);
		  \draw[double,thick] (0.5,-2.2) to node[above] {$x_n$} (1.8,-2.2);
		  \draw[double,thick] (1.8,-2.2) to (1.8,1.35);
		  \draw[double,thick] (3.5,-1.6) to node[above] {$b_1$} (4.8,-1.6);
		  \draw[double,thick] (4.8,-1.6) to (4.8,1.35);
		  \draw[double,thick] (5.2,1.3) to (5.4,1.3);
		  \draw[double,thick] (5.4,1.3) to (5.4,2.2);
		  \draw[double,thick] (5.4,2.2) to (5.5,2.2);
		  
		  \draw[double,thick] (2.2,1.3) to (2.4,1.3);
		  \draw[double,thick] (2.4,1.3) to (2.4,1.7);
		  \draw[double,thick] (2.4,1.7) to (2.5,1.7);
		  

		  
		  \draw[double,thick] (0.5,2.2) to node[above] {$y_1$} (2,2.2);
		  \draw[double,thick] (2,2.2) to (2,-2.2);
		  \draw[double,thick] (2,-2.2) to (2.5,-2.2);
		  
		  \draw[double,thick] (3.5,1.6) to node[above] {$a_n$} (5,1.6);
		  \draw[double,thick] (5,1.6) to (5,-2.2);
		  \draw[double,thick] (5,-2.2) to (5.5,-2.2);
      
      \node at (2.18,1.8) (input2) {};
      \node at (3.5,2) (block2) {};
      
      \node at (5.18,0.6) (input3) {};
      \node at (3.5,2) (block3) {};
      
    % Calculate branch point coordinate
    \path (input2) -- coordinate (branch) (block2);
    \path (input3) -- coordinate (branch) (block3);
    

    \draw[double,thick] (branch) node[branch] {}{ % draw branch junction
             (branch -| input2) arc(0:180:\radius)
            
        };

    \draw[double,thick] (branch) node[branch] {}{ % draw branch junction
             (branch -| input3) arc(0:180:\radius)
            
        };        
        
\end{tikzpicture}
\end{center}

\end{document}
